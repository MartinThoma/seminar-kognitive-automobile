%!TEX root = sicherheit-martin-thoma.tex
\section{Angriffsarten und -ziele}

In der IT-Sicherheit unterscheidet man zwischen Angriffsvektor und
\textit{Angriffsziel}. Als \textit{Angriffsvektor} wird der Weg sowie die Art
und Weise bezeichent, wie ein Angriff durchgeführt wird\cite{Sambleben2012,Metasploit2012}.
Außerdem unterscheidet man Sicherheitslücken und Exploits. Eine
Sicherheitslücke stellt eine Schwachstelle in einem System dar. Wenn diese
Lücke genutzt werden kann um Schaden zuzufügen, dann spricht man von einem
\textit{Exploit}.


\subsection{Angriffsziele}
Im Automobilbereich sind verschiedene Motive vorstellbar, die einen Angriff
auf ein Auto erfordern

\begin{itemize}
    \item Beschaffung von Persönlichen Informationen
    \item Diebstahl
    \item Zerstörung
    \item Mord
\end{itemize}

Je nach Motivation sind verschiedene Ziele im Auto vorstellbar:

\begin{itemize}
    \item Positionsinformationen
    \item Meta-Informationen wie Modell, gefahrene Kilometer
    \item Öffnen der Türen
    \item Beschleunigen, Bremsen
\end{itemize}


\subsection{Angriffsvektoren}
Jede Schnittstelle zum Auto stellt einen möglichen Angriffsvektor dar.
Allerdings stellen nicht nur Empfänger mögliche Angriffsvektoren dar, sondern
in bezug auf die Privatsphäre auch Sender.

\begin{itemize}
    \item OBD-Schnittstelle
    \item TPMS
    \item eCall
\end{itemize}

\subsection{Reverse-Engineering}
\begin{itemize}
    \item CAN-Bus
    \item OBD-Diagnosegerät
    \item TPMS
    \item eCall?
\end{itemize}