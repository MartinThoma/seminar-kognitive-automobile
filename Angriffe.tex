%!TEX root = sicherheit-martin-thoma.tex
\section{Angriffsarten und -ziele}\label{sec:attack}

In der IT-Sicherheit unterscheidet man zwischen Angriffsvektor und
\textit{Angriffsziel}. Als \textit{Angriffsvektor} wird der Weg sowie die Art
und Weise bezeichent, wie ein Angriff durchgeführt wird\cite{Sambleben2012,Metasploit2012}.
Außerdem unterscheidet man Sicherheitslücken und Exploits. Eine
Sicherheitslücke stellt eine Schwachstelle in einem System dar. Wenn diese
Lücke genutzt werden kann um Schaden zuzufügen, dann spricht man von einem
\textit{Exploit}.


\subsection{Angriffsziele}
Im Automobilbereich sind verschiedene Motive wie Beschaffung von persönlichen
Informationen, Diebstahl, Zerstörung und Mord, vorstellbar, die einen Angriff
auf ein Auto erfordern. Je nach Motivation sind verschiedene Ziele im Auto für
den Angreifer von Interesse:

\begin{itemize}
    \item Positionsinformationen
    \item Meta-Informationen wie Modell oder gefahrene Kilometer
    \item Öffnen der Türen
    \item Beschleunigen oder Bremsen
\end{itemize}

Im Folgenden wird beschrieben, wie diese Daten abgegriffen werden können.


\subsection{Angriffsvektoren}
Jeder Empfänger stellt als Schnittstelle zum Auto einen möglichen
Angriffsvektor dar. Allerdings sind in Bezug auf die Privatsphäre auch Sender
mögliche Angriffswege. Dabei ist insbesondere die OBD-Schnittstelle, TPMS und
eCall zu nennen.

\subsection{Reverse-Engineering}
Reverse-Engineering bezeichnet den Vorgang ein System nachzuentwickeln. Im
Bezug auf Computersicherheit wird dies von Angreifern gemacht um die genaue
Funktionsweise des Originals nachvollziehen und mögliche Fehler in der
Entwicklung oder sogar im Design zu entdecken. Reverse-Engineering kann nicht
verhindert, aber erschwert werden.

Die Sicherheit keines Computersystems sollte ausschließlich auf der
Geheimhaltung der Implementierung beruhen \cite{ServerSecurity2008}, sie kann
aber eine weitere Sicherungsschicht sein. Diese kann es dem Angreifer
erschweren, den eigentlichen Angriff durchzuführen oder Fehler in der
Implementierung verdecken.

Aus diesem Grund sollte das Reverse-Engineering schwer sein. Insbesondere
sollten sog. Debugging-Symbole, also Informationen welche im der Binärdatei
hinterlegt werden um Fehler aufzuspüren, nicht in verkauften Autos sein. Des
Weiteren sollten Fehlernachricht aus dem Produktivcode entfernt werden. Beides
wurde in \cite{Checkoway2011} in einem untersuchtem Auto im Code von ECUs
gefunden gefunden.


\subsection{Angriffszenarien}