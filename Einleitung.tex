%!TEX root = sicherheit-martin-thoma.tex

\section{Einleitung}
Kognitive Automobile sind, im Gegensatz zu klassichen Automobilen, in der Lage
ihre Umwelt und sich selbst wahrzunehmen und dem Fahrer zu assistieren oder
auch teil- bzw. vollautonom zu fahren. Diese Systeme benötigen Zugriff auf
Sensoren und Aktoren, um ihre Aufgabe zu erfüllen. So benötigt ein Auto mit
Antiblockiersystem beispielsweise die Drehzahl an jedem Reifen und die
Möglichkeit die Bremsen zu beeinflussen; für Einparkhilfen werden Sensoren
benötigt, welche die Distanz zu Hindernissen wahrnehmen sowie Aktoren, die das
Auto lenken und beschleunigen können. Weitere dieser Systeme sind
Spurhalteassistenz, Spurwechselassistenz und Fernlichtassistenz.

Als immer mehr elektronische Systeme in Autos verbaut wurden, die teilweise
redundante Aufgaben erledigt haben, wurde der CAN-Bus
entwickelt\cite{Kiencke1986}. Über ihn kommunizieren elektronische
Steuergeräte, sog. \textit{ECUs} (engl. \textit{electronic control units}).
Diese werden beispielsweise für ABS und ESP eingesetzt. % das \gls{abs} und das \gls{esp}

Eine Reihe von elektronischen Systemen wurde zum Diebstahlschutz entwickelt
\cite{Song2008,Turner1999,Hwang1997}. Allerdings passen sich auch Diebe an die
modernen Gegebenheiten an.

Der folgende Abschnitt geht auf Standards wie den CAN-Bus und Verordnungen, die
in der Europäischen Union gültig sind, ein. In \cref{sec:attack} werden
Angriffsziele und Grundlagen zu den Angriffen erklärt, sodass in
\cref{sec:defense} mögliche Verteidigungsmaßnahmen erläutert werden können.