%!TEX root = sicherheit-martin-thoma.tex

\section{Einleitung}
Kognitive Automobile sind, im Gegensatz zu klassichen Automobilen, in der Lage
ihre Umwelt und sich selbst wahrzunehmen und dem Fahrer zu assistieren. So
benötigt ein Auto mit Antiblockiersystem beispielsweise die Drehzahl an jedem
Reifen; für Einparkhilfen werden Sensoren benötigt, welche die Distanz zu
Hindernissen wahrnehmen sowie Aktoren, die das Auto lenken und beschleunigen
könne. Weitere dieser System sind Spurhalteassistenz, Spurwechselassistenz und
Fernlichtassistenz. Um die Funktionalität dieser Assistenzsysteme bereitstellen
zu können, muss das Auto die Umwelt wahrnehmen und -- in festgelegten Grenzen
-- autonom agieren können.

Als immer mehr elektronische Systeme in Autos verbaut wurden, die teilweise
redundante Aufgaben erledigt haben, wurde der CAN-Bus
entwickelt\cite{Kiencke1986}. Dieser wurde in ISO 11898 international
standardisiert.