%!TEX root = ../sicherheit-martin-thoma.tex
\chapter{Angriffe}\label{ch:attack}

Eine Reihe von elektronischen Systemen wurde zum Diebstahlschutz entwickelt
\cite{Song2008,Turner1999,Hwang1997}. Allerdings passen sich auch Diebe an die
modernen Gegebenheiten, insbesondere Funkschlüssel \cite{Lee2014}, an. Außerdem
gehen diese Systeme von einem klassischen Angreifer aus, der sich
ausschließlich auf der Hardware-Ebene bewegt.

Im Folgenden werden zunächst Möglichkeiten von netzwerk-internen Angreifer,
d.h. Angreifern welche physischen Zugang zum CAN-Bus haben, aufgelistet. Es
folgt eine Beschreibung wie Angreifer ohne direkten physischen Zugriff auf das
Auto sich mit dem CAN-Bus verbinden können.


\section{CAN-interne Angriffe}
Koscher~et~al.~haben in \cite{Koscher2010} zwei nicht näher spezifizierte Autos
der selben Marke und des selben Modells untersucht. Sie waren in der Lage über
den CAN-Bus etliche Funktionen des Autos, unabhängig vom Fahrer, zu
manipulieren. Das beinhaltet: deaktivieren und aktivieren der Bremsen, stoppen
des Motors, steuern der Klimaanlage, Heizung, Lichter, Manipulation der
Anzeigen im Kombiinstrument sowie das Öffnen und Schließen der Schlösser. Durch
moderne Systeme wie eCall kann der Angreifer sich sogar einen
Kommunikationskanal zu dem Auto aufbauen. Dies setzt allerdings voraus, dass
der Angreifer sich bereits im Auto-Internen Netzwerk befindet.

\section{CAN-externe Angriffe}
In~\cite{Checkoway2011} wurde an einer Mittelklasselimosine mit
Standardkomponenten gezeigt, dass der Zugang zum auto-internen Netzwerk über
eine Vielzahl an Komponenten erfolgen kann. So haben Checkoway~et~al.
CD-Spieler, Bluetooth und den OBD-Zugang als mögliche Angriffsvektoren
identifiziert.

Bei dem Angriff über den Media Player haben Checkoway~et~al. die Tatsache
genutzt, dass dieser am CAN-Bus angeschlossen ist und die Software des
Mediaplayers über eine CD mit einem bestimmten Namen und Dateiformat
aktualisiert werden kann. Außerdem wurde ein Fehler beim abspielen der
Audio-Dateien genutzt um einen Buffer Overflow zu erzeugen. Es wurde gezeigt,
dass dieser genutzt werden kann um die Software des Media-Players zu
aktualisieren. Dafür muss nur eine modifizierte Audio-Datei, welche immer noch
abspielbar ist, auf der CD sein.

Der von Checkoway~et~al. durchgeführte Angriff via Bluetooth benötigt ein
mit dem Media Player verbundenes Gerät. Dieses nutzt dann unüberprüfte
\verb+strcpy+-Befehle bei der Bluetooth-Konfiguration aus um beliebigen Code
auf der Telematik-Einheit des Autos ausführen zu können. Daher ist das
Smartphone des Autobesitzers ein Angriffsvektor.

Die Bluetooth-Verbindung kann jedoch auch ohne ein verbundenes Gerät für
Angriffe genutzt werden. Dazu muss der Angreifer genügend Zeit in der Nähe des
Autos verbringen um die Bluetooth-MAC-Adresse zu erfahren. Damit kann er ein
Anfrage zum Verbindungsaufbau starten. Diese müsste der Benutzer normalerweise
mit der Eingabe einer PIN bestätigen. Checkoway~et~al. haben für ein Auto
gezeigt, dass die Benutzerinteraktion nicht nötig it und die PIN via
Brute-Force, also das Ausprobieren aller Möglichkeiten, innerhalb von
10~Stunden gefunden werden kann. Allerdings kann dieser Angriff parallel
ausgeführt werden. Es ist also beispielsweise möglich diesen Angriff für alle
Autos in einem Parkhaus durchzuführen.

Die standardisierte und von Automechanikern zu Diagnosezwecken genutzte
OBD-Schnittstelle stellt einen weiteren Angriffspunkt dar. Für die
verschiedenen Marken gibt es Diagnosewerkzeuge, wie z.B. NGS für Ford,
Consult~II für Nissan und der Diagnostic~Tester von Toyota. Diese dedizierten
Diagnosegeräte werden allerdings über PCs mit aktualisierungen versorgt.
Modernere Diagnosewerkzeuge sind nicht mehr bei der Diagnose vom PC getrennt,
sondern werden direkt, über ein Kabel, W-LAN oder Bluetooth, mit einem PC
verbunden. Daher stellt die Diagnose- und Aktualisierungstätigkeit von
Automechanikern einen weiteren Angriffsvektor dar. Wenn der Mechaniker ein
Diagnosegerät benutzt, welches ein W-LAN aufbaut, so können Angreifer sich
mit diesem verbinden und selbst Aktualisierungen durchführen. Außerdem wurde
von Checkoway~et~al. gezeigt, dass auch das Diagnosegerät selbst so manipuliert
werden kann, dass es automatisch die gewünschten Angriffe ausführt.

Wie in \cref{sec:standards} beschrieben wird eCall-System ab 2018 in Europa
verpflichtend eingeführt. Dieses nutzt das Mobilfunknetz zur Kommunikation.
Checkoway~et~al. haben gezeigt, dass Telematik-Einheiten von außerhalb des
Autos angewählt werden und die Software auf diese Weise aus beliebigen
Entfernungen aktualisiert werden kann. Dazu wurden zahlreiche Schwachstellen
der Telematik-Einheit von Airbiquitys Modem aqLink genutzt. Dieses Modem wird
unter anderem von BMW und Ford eingesetzt \cite{AirbiquityBMW,AirbiquityFord}.

Ein Angriff auf die Privatsphäre ist mit TPMS möglich. In~\cite{Rouf2010} wurde
gezeigt, dass TPMS-Signale zur Identifikation von Autos genutzt werden können.
Die Identifikation eines vorbeifahrenden Autos ist aus bis zu \SI{40}{\meter}
Entfernung möglich.

Genauso stellt das Mikrofon, welches wegen eCall ab 2018 in jedem Auto sein
muss, eine Möglichkeit zum Angriff auf die Privatsphäre dar.


% \section{Reverse-Engineering}
% Reverse-Engineering bezeichnet den Vorgang ein System nachzuentwickeln. Im
% Bezug auf Computersicherheit wird dies von Angreifern gemacht um die genaue
% Funktionsweise des Originals nachvollziehen und mögliche Fehler in der
% Entwicklung oder sogar im Design zu entdecken. Reverse-Engineering kann nicht
% verhindert, aber erschwert werden.

% Die Sicherheit keines Computersystems sollte ausschließlich auf der
% Geheimhaltung der Implementierung beruhen \cite{ServerSecurity2008}, sie kann
% aber eine weitere Sicherungsschicht sein. Diese kann es dem Angreifer
% erschweren, den eigentlichen Angriff durchzuführen oder Fehler in der
% Implementierung verdecken.

% Aus diesem Grund sollte das Reverse-Engineering schwer sein. Insbesondere
% sollten sog. Debugging-Symbole, also Informationen welche im der Binärdatei
% hinterlegt werden um Fehler aufzuspüren, nicht in verkauften Autos sein. Des
% Weiteren sollten Fehlernachricht aus dem Produktivcode entfernt werden. Beides
% wurde in \cite{Checkoway2011} in einem untersuchtem Auto im Code von ECUs
% gefunden gefunden.
