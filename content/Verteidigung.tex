%!TEX root = ../booka4.tex
\chapter{Verteidigungsmaßnahmen}\label{ch:defense}
Der CAN-Bus ist eine große Schwachstelle der IT-Sicherheit in Autos. Über ihn
müssen viele ECUs kommunizieren und einige, wie das Autoradio, werden nicht als
Sicherheitskritsch wahrgenommen.

Daher ist es wichtig die Nachrichten, welche über den CAN-Bus empfangen werden,
zu filtern. Die Informationen müssen auf Plausibilität geprüft werden.

Alle von Checkoway~et~al. beschriebenen Angriffe basieren zum einen auf
Reverse-Engineering, also der Rekonstruktion der Software-Systeme und
Protokolle, zum anderen auf Fehlern in der Software. Das Reverse-Engineering
wurde in einigen Fällen laut Checkoway~et~al. stark vereinfacht, da
Debugging-Symbole in der Software war. Diese können und sollten einfach
entfernt werden.

Außerdem sollten laut Checkoway~et~al. die Diagnosegeräte Authentifizierung und
Verschlüsselung wie beispielsweise OpenSSL nutzen.

Gegen Buffer-Overflow-Angriffe können zum einen Sprachen wie Java oder Rust
verwendet werden, welche die Einhaltung der Bereichsgrenzen automatisch
überprüfen. Des weiteren kann anstelle der C-Funktion \verb+strcpy()+ die
Funktion \verb+strncpy()+ verwendet werden, welche die Anzahl der zu
schreibenden Zeichen begrenzt \cite{Eckert2012}. Ein weiteres Konzept zum
Schutz vor Buffer-Overflow-Angriffen sind Stack Cookies \cite{Bray2002}.

Code Reviews können auch solche Sicherheitslücken aufdecken \cite{Howard2006}.
Code Reviews können teilweise automatisch mit Werkzeugen zur statischen Code
Analyse durchgeführt werden \cite{McGraw2008}.