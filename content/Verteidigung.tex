%!TEX root = ../booka4.tex
\chapter{Verteidigungsmaßnahmen}\label{ch:defense}
Wie bereits in \cref{sec:sicherheitslage} beschrieben, sind die Arten der
Angriffe nicht neu. Daher sind auch die Verteidigungsmaßnahmen nicht
spezifisch für den Automobilbereich, sondern allgemeiner softwaretechnischer
Art.

Es gibt etliche Automobilhersteller, -marken und -modelle. Für viele Modelle
gibt es unterschiedliche Konfigurationen und wiederum zahlreiche Optionen für
Zubehör wie beispielsweise das Autoradio oder Navigationssysteme. Dies macht
allgemeine Aussagen über die Sicherheit und Verteidigungsmaßnahmen von
Automobilen schwierig. Allerdings stellen Standards und Verordnungen sicher,
dass Teile der relevanten Infrastruktur in Automobilen gleich sind, sodass
Angreifer diese fahrzeugübergreifend nutzen können.

Der CAN-Bus ist eine große Schwachstelle der IT-Sicherheit in Autos. Über ihn
müssen viele ECUs kommunizieren und einige, wie das Autoradio, werden nicht als
Sicherheitskritsch wahrgenommen. Gleichzeitig sind sicherheitskritische ECUs an
dem selben CAN-Bus angeschlossen. Daher ist es wichtig die Nachrichten, welche
über den CAN-Bus empfangen werden, zu filtern. Die Informationen müssen auf
Plausibilität geprüft werden.

Alle von Checkoway~et~al. beschriebenen Angriffe basieren zum einen auf
Reverse-Engineering, also der Rekonstruktion der Software-Systeme und
Protokolle, zum anderen auf Fehlern in der Software. Das Reverse-Engineering
wurde in einigen Fällen laut Checkoway~et~al. stark vereinfacht, da
Debugging-Symbole in der Software waren. Diese können und sollten
entfernt werden.

Außerdem sollten laut Checkoway~et~al. die Diagnosegeräte Authentifizierung und
Verschlüsselung wie beispielsweise OpenSSL nutzen.

Gegen Buffer-Overflow-Angriffe können zum einen Sprachen wie Java oder Rust
verwendet werden, welche die Einhaltung der Bereichsgrenzen automatisch
überprüfen. Des weiteren kann anstelle der C-Funktion \verb+strcpy()+ die
Funktion \verb+strncpy()+ verwendet werden, welche die Anzahl der zu
schreibenden Zeichen begrenzt \cite{Eckert2012}. Ein weiteres Konzept zum
Schutz vor Buffer-Overflow-Angriffen sind Stack Cookies \cite{Bray2002}.

Code Reviews können auch solche Sicherheitslücken aufdecken \cite{Howard2006}.
Code Reviews können teilweise automatisch mit Werkzeugen zur statischen Code
Analyse durchgeführt werden \cite{McGraw2008}.

Eine weiterer wichtiger Stützpfeiler für sichere Software sind schnell
ausgelieferte Sicherheitsaktualisierungen. Dazu gehört laut \cite{Mahaffey2015}
unter anderem ein System zum mobilen versenden von Aktualisierungen an Autos
mit Mobilfunkverbindung.