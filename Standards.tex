%!TEX root = sicherheit-martin-thoma.tex
\section{Standards und Verordnungen}\label{sec:standards}
Für den Automobilbereich existieren viele Standards und Verordnungen. In diesem
Abschnitt wird eine Auswahl vorgestellt, die Fahrzeuge der Klassen M$_1$ und
N$_1$ betrifft. Das sind Fahrzeuge zur Personenbeförderung
\enquote{mit mindestens vier Rädern und höchstens acht Sitzplätzen außer dem Fahrersitz}
sowie \enquote{für die Güterbeförderung ausgelegte und gebaute Kraftfahrzeuge
mit einer zulässigen Gesamtmasse von 3,5 Tonnen}\cite{Richtlinie70/156/EWG:Fahrzeugklassen}.

In der EU wurde mit \cite{EUDirective98/69/EC} die OBD-Schnittstelle
verpflichtend für Fahrzeuge der Klasse M$_1$ und N$_1$ mit Fremdzündungsmotor
ab 1.~Januar 2004. Die EU-Direktive führt weiter die in der ISO DIS 15031-6
Norm aufgeführten Fehlercodes als Minimalstandard ein. Diese müssen
\enquote{für genormte Diagnosegeräte [\dots] uneingeschränkt zugänglich sein}.
Außerdem muss die Schnittstelle im Auto so verbaut werden, dass sie
\enquote{für das Servicepersonal leicht zugänglich, zugleich aber vor
unbefugten Eingriffen durch nichtqualifizierte Personen geschützt ist}.

Um die Daten bereitzustellen, werden verschiedene elektronische Komponenten
über den CAN-Bus vernetzt. Dieser ist in ISO 11898 genormt.

Weiterhin wurde in der EU mit \cite{EURegulation661/2009} beschlossen, dass ab
1.~November 2012 alle PKWs für Neuzulassungen ein System zur
Reifendrucküberwachung (engl. \textit{tire pressure monitoring system}, kurz
\textit{TPMS}) besitzen müssen. Ab 1.~November 2014 müssen alle Neuwagen ein
solches System besitzen.

Mit \cite{EURegulation2015/ecall} wird für Fahrzeuge, die ab dem 31.~März 2018
gebaut werden das eCall-System, ein elektronisches Notrufsystem, verpflichtend.
Dabei müssen dem eCall-System \enquote{präzise[-] und verlässliche[-]
Positionsdaten} zur Verfügung stehen, welche über das globales
Satellitennavigationssystem Galileo und dem Erweiterungssystem EGNOS geschehen
soll. eCall soll über öffentliche Mobilfunknetze eine \enquote{Tonverbindung
zwischen den Fahrzeuginsassen und einer eCall-Notrufabfragestelle} herstellen
können. Außerdem muss ein Mindestdatensatz übermittelt werden, welcher in
DIN EN 15722:2011 geregelt ist. Diese Funktionen müssen im Fall eines schweren
Unfalls automatisch durchgeführt werden können.