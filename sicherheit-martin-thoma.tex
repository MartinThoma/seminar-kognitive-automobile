\documentclass{IEEEtran}

%---- Sonderzeichen-------%
\usepackage[utf8]{inputenc} % this is needed for umlauts
\usepackage[ngerman]{babel} % this is needed for umlauts
\usepackage[T1]{fontenc}    % this is needed for correct output of umlauts in pdf
%---- Codierung----%
\usepackage{graphicx}
\usepackage{url}
\usepackage{llncsdoc}
%----- Mathematischer Zeichenvorrat---%
\usepackage{amsmath}
\usepackage{amssymb}
\usepackage{enumerate}
% fuer die aktuelle Zeit
\usepackage{scrtime}
\usepackage{listings}
\usepackage{hyperref}
\usepackage{cite}
\usepackage{parskip}
\usepackage[framed,amsmath,thmmarks,hyperref]{ntheorem}
\usepackage{algorithm}
\usepackage[noend]{algpseudocode}
\usepackage{csquotes}
\usepackage[colorinlistoftodos]{todonotes}
\usepackage{subfig}         % multiple figures in one
\usepackage{caption}
\usepackage{tikz}
\usepackage{enumitem}
\usepackage[german,nameinlink]{cleveref}
\usepackage{braket}
\allowdisplaybreaks
\usetikzlibrary{backgrounds}
\usepackage[binary-units=true]{siunitx}
\usepackage{mystyle}

\setcounter{tocdepth}{3}
\setcounter{secnumdepth}{3}

\hypersetup{
  pdftitle    = {Sicherheit im Automobilbereich},
  pdfauthor   = {Martin Thoma},
  pdfkeywords = {Sicherheit}
}

\begin{document}

%\mainmatter
\title{Sicherheit im Automobilbereich}
%\titlerunning{}
\author{\IEEEauthorblockN{Martin Thoma}
\IEEEauthorblockA{Karlsruhe Institute of Technology\\
Email: info@martin-thoma.de}}
%\authorrunning{Seminar Kognitive Automobile}
%\institute{Betreuer: Ralf Kohlhaas}
\date{24.04.2015}
\maketitle

\begin{abstract}%
In dieser Arbeit geht es um Sicherheit im Automobilbereich. 
Lorem ipsum dolor sit amet, consectetur adipiscing elit. Aliquam fringilla risus ut purus tincidunt sodales ac ut est. Ut eget tempus eros. Interdum et malesuada fames ac ante ipsum primis in faucibus. Sed et tempor nisl, et semper sapien. Curabitur non ex mi. Donec nec purus vel dolor porttitor rutrum malesuada sed mi. Morbi mollis tincidunt pulvinar. Pellentesque suscipit lacus magna, id cursus felis gravida ut. Vivamus lobortis dictum rhoncus. Proin tincidunt volutpat tortor, vel volutpat turpis porttitor eu. Praesent id risus vitae orci auctor suscipit.

\textbf{Keywords:} Sicherheit

\end{abstract}


TODO \cite{Koscher2010}

Lorem ipsum dolor sit amet, consectetur adipiscing elit. Aliquam fringilla risus ut purus tincidunt sodales ac ut est. Ut eget tempus eros. Interdum et malesuada fames ac ante ipsum primis in faucibus. Sed et tempor nisl, et semper sapien. Curabitur non ex mi. Donec nec purus vel dolor porttitor rutrum malesuada sed mi. Morbi mollis tincidunt pulvinar. Pellentesque suscipit lacus magna, id cursus felis gravida ut. Vivamus lobortis dictum rhoncus. Proin tincidunt volutpat tortor, vel volutpat turpis porttitor eu. Praesent id risus vitae orci auctor suscipit.

Donec nec laoreet nisi. In tellus diam, euismod a molestie non, tincidunt in risus. Curabitur condimentum accumsan arcu, nec accumsan libero mollis congue. Praesent ac sollicitudin ligula, a pulvinar sapien. Proin eu auctor lacus. Donec et leo porttitor, pretium lacus eget, facilisis mauris. Quisque ut faucibus dui. Maecenas tempor urna erat, in mattis mauris ullamcorper ac. Cras facilisis id est vitae auctor. Phasellus sollicitudin erat neque, ac molestie justo venenatis a. Cras velit magna, tincidunt ut condimentum id, commodo id ipsum. Ut aliquam tempus elit ac malesuada. Ut pellentesque est nec nunc cursus imperdiet. Integer feugiat, libero quis efficitur eleifend, risus justo sollicitudin dui, ac dictum ipsum libero et sapien. Quisque quam metus, gravida a elit ut, fermentum sodales nisi. Nulla aliquet dolor in massa feugiat sodales.

Cras lacinia sem at lectus pulvinar, sed iaculis justo mollis. Duis vel ligula est. Vivamus ac porta enim. Ut sit amet euismod eros. Phasellus auctor nulla venenatis ultricies pretium. Fusce eu pellentesque erat, nec eleifend elit. Vivamus volutpat lectus risus, at faucibus ipsum rutrum eget. Sed venenatis feugiat ex sed efficitur. Nulla tincidunt mollis turpis dignissim sodales.

Mauris et odio condimentum, bibendum mi at, posuere augue. Praesent posuere auctor tristique. Quisque accumsan est dui, id gravida orci iaculis eu. Praesent tincidunt tristique lorem non blandit. Mauris convallis commodo justo quis tristique. Integer ultrices quam nec est efficitur scelerisque nec et mauris. Suspendisse potenti. Proin pharetra posuere viverra.
%!TEX root = sicherheit-martin-thoma.tex
\section{Standards und Verordnungen}\label{sec:standards}
Für den Automobilbereich existieren viele Standards und Verordnungen. In diesem
Abschnitt wird eine Auswahl vorgestellt, die Fahrzeuge der Klassen M$_1$ und
N$_1$ betrifft. Das sind Fahrzeuge zur Personenbeförderung
\enquote{mit mindestens vier Rädern und höchstens acht Sitzplätzen außer dem Fahrersitz}
sowie \enquote{für die Güterbeförderung ausgelegte und gebaute Kraftfahrzeuge
mit einer zulässigen Gesamtmasse von 3,5 Tonnen}\cite{Richtlinie70/156/EWG:Fahrzeugklassen}.

In der EU wurde mit \cite{EUDirective98/69/EC} die OBD-Schnittstelle
verpflichtend für Fahrzeuge der Klasse M$_1$ und N$_1$ mit Fremdzündungsmotor
ab 1.~Januar 2004. Die EU-Direktive führt weiter die in der ISO DIS 15031-6
Norm aufgeführten Fehlercodes als Minimalstandard ein. Diese müssen
\enquote{für genormte Diagnosegeräte [\dots] uneingeschränkt zugänglich sein}.
Außerdem muss die Schnittstelle im Auto so verbaut werden, dass sie
\enquote{für das Servicepersonal leicht zugänglich, zugleich aber vor
unbefugten Eingriffen durch nichtqualifizierte Personen geschützt ist}.

Um die Daten bereitzustellen, werden verschiedene elektronische Komponenten
über den CAN-Bus vernetzt. Dieser ist in ISO 11898 genormt.

Weiterhin wurde in der EU mit \cite{EURegulation661/2009} beschlossen, dass ab
1.~November 2012 alle PKWs für Neuzulassungen ein System zur
Reifendrucküberwachung (engl. \textit{tire pressure monitoring system}, kurz
\textit{TPMS}) besitzen müssen. Ab 1.~November 2014 müssen alle Neuwagen ein
solches System besitzen.

Mit \cite{EURegulation2015/ecall} wird für Fahrzeuge, die ab dem 31.~März 2018
gebaut werden das eCall-System, ein elektronisches Notrufsystem, verpflichtend.
Dabei müssen dem eCall-System \enquote{präzise[-] und verlässliche[-]
Positionsdaten} zur Verfügung stehen, welche über das globales
Satellitennavigationssystem Galileo und dem Erweiterungssystem EGNOS geschehen
soll. eCall soll über öffentliche Mobilfunknetze eine \enquote{Tonverbindung
zwischen den Fahrzeuginsassen und einer eCall-Notrufabfragestelle} herstellen
können. Außerdem muss ein Mindestdatensatz übermittelt werden, welcher in
DIN EN 15722:2011 geregelt ist. Diese Funktionen müssen im Fall eines schweren
Unfalls automatisch durchgeführt werden können.
%!TEX root = sicherheit-martin-thoma.tex
\section{Angriffsarten und -ziele}

\begin{itemize}
    \item CAN-Bus
    \item OBD-Diagnosegerät
    \item TPMS
    \item eCall?
\end{itemize}
%!TEX root = sicherheit-martin-thoma.tex
\section{Verteidigungsmaßnahmen}
Wie in den vorherigen Abschnitten beschrieben wurde, ist der CAN-Bus eine große
Schwachstelle der IT-Sicherheit in Autos. Über ihn müssen viele ECUs
kommunizieren und einige, wie das Autoradio, werden nicht als
Sicherheitskritsch wahrgenommen.

Daher ist es wichtig die Nachrichten, welche über den CAN-Bus empfangen werden,
zu filtern. Die Informationen müssen auf Plausibilität geprüft werden.

% Normaler LNCS Zitierstil
%\bibliographystyle{splncs}
\bibliographystyle{itmalpha}
% TODO: Ändern der folgenden Zeile, damit die .bib-Datei gefunden wird
\bibliography{literatur}

\end{document}

