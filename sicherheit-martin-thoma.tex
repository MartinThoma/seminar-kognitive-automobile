\documentclass{IEEEtran}

%---- Sonderzeichen-------%
\usepackage[utf8]{inputenc} % this is needed for umlauts
\usepackage[ngerman]{babel} % this is needed for umlauts
\usepackage[T1]{fontenc}    % this is needed for correct output of umlauts in pdf
%---- Codierung----%
\usepackage{graphicx}
\usepackage{url}
\usepackage{llncsdoc}
%----- Mathematischer Zeichenvorrat---%
\usepackage{amsmath}
\usepackage{amssymb}
\usepackage{enumerate}
% fuer die aktuelle Zeit
\usepackage{scrtime}
\usepackage{listings}
\usepackage{hyperref}
\usepackage{cite}
\usepackage{parskip}
\usepackage[framed,amsmath,thmmarks,hyperref]{ntheorem}
\usepackage{algorithm}
\usepackage[noend]{algpseudocode}
\usepackage{csquotes}
\usepackage[colorinlistoftodos]{todonotes}
\usepackage{subfig}         % multiple figures in one
\usepackage{caption}
\usepackage{tikz}
\usepackage{enumitem}
\usepackage[german,nameinlink]{cleveref}
\usepackage{braket}
\allowdisplaybreaks
\usetikzlibrary{backgrounds}
\usepackage[binary-units=true]{siunitx}
\usepackage{mystyle}

\setcounter{tocdepth}{3}
\setcounter{secnumdepth}{3}

\hypersetup{ 
  pdftitle    = {Sicherheit im Automobilbereich},
  pdfauthor   = {Martin Thoma}, 
  pdfkeywords = {Sicherheit}
}

\begin{document}

%\mainmatter
\title{Sicherheit im Automobilbereich}
%\titlerunning{}
\author{Martin Thoma}
%\authorrunning{Seminar Kognitive Automobile}
%\institute{Betreuer: Christopher Oßner}
\date{24.04.2015}
\maketitle

\begin{abstract}%
Moderne Automobile verfügen über eine Vielzahl von Assistenz- und
Fahrsicherheitssystemen. Diese Systeme haben Schnittstellen, welche das Ziel
von Angriffen sein können. In dieser Seminararbeit wird der aktuelle Stand der
IT-Sicherheit kognitiver Automobilie untersucht. Dabei wird auf mögliche
Angriffsvektoren und Ziele sowie Möglichkeiten zum Schutz eingegangen.

\textbf{Keywords:} Sicherheit

\end{abstract}

\section{Einleitung}
%!TEX root = sicherheit-martin-thoma.tex

\section{Einleitung}
Kognitive Automobile sind, im Gegensatz zu klassichen Automobilen, in der Lage
ihre Umwelt und sich selbst wahrzunehmen und dem Fahrer zu assistieren oder
auch teil- bzw. vollautonom zu fahren. Diese Systeme benötigen Zugriff auf
Sensoren und Aktoren, um ihre Aufgabe zu erfüllen. So benötigt ein Auto mit
Antiblockiersystem beispielsweise die Drehzahl an jedem Reifen und die
Möglichkeit die Bremsen zu beeinflussen; für Einparkhilfen werden Sensoren
benötigt, welche die Distanz zu Hindernissen wahrnehmen sowie Aktoren, die das
Auto lenken und beschleunigen können. Weitere dieser Systeme sind
Spurhalteassistenz, Spurwechselassistenz und Fernlichtassistenz.

Als immer mehr elektronische Systeme in Autos verbaut wurden, die teilweise
sich überschneidende Aufgaben erledigt haben, wurde der CAN-Bus
entwickelt\cite{Kiencke1986}. Über ihn kommunizieren elektronische
Steuergeräte, sog. \textit{ECUs} (engl. \textit{electronic control units}).
Diese werden beispielsweise für ABS und ESP eingesetzt. % das \gls{abs} und das \gls{esp}

Der folgende Abschnitt geht auf Standards wie den CAN-Bus und Verordnungen, die
in der Europäischen Union gültig sind, ein. In \cref{sec:attack} werden
Angriffsziele und Grundlagen zu den Angriffen erklärt, sodass in
\cref{sec:defense} mögliche Verteidigungsmaßnahmen erläutert werden können.

\section{Related Work}
Lorem ipsum dolor sit amet, consectetur adipiscing elit. Aliquam fringilla risus ut purus tincidunt sodales ac ut est. Ut eget tempus eros. Interdum et malesuada fames ac ante ipsum primis in faucibus. Sed et tempor nisl, et semper sapien. Curabitur non ex mi. Donec nec purus vel dolor porttitor rutrum malesuada sed mi. Morbi mollis tincidunt pulvinar. Pellentesque suscipit lacus magna, id cursus felis gravida ut. Vivamus lobortis dictum rhoncus. Proin tincidunt volutpat tortor, vel volutpat turpis porttitor eu. Praesent id risus vitae orci auctor suscipit.

Donec nec laoreet nisi. In tellus diam, euismod a molestie non, tincidunt in risus. Curabitur condimentum accumsan arcu, nec accumsan libero mollis congue. Praesent ac sollicitudin ligula, a pulvinar sapien. Proin eu auctor lacus. Donec et leo porttitor, pretium lacus eget, facilisis mauris. Quisque ut faucibus dui. Maecenas tempor urna erat, in mattis mauris ullamcorper ac. Cras facilisis id est vitae auctor. Phasellus sollicitudin erat neque, ac molestie justo venenatis a. Cras velit magna, tincidunt ut condimentum id, commodo id ipsum. Ut aliquam tempus elit ac malesuada. Ut pellentesque est nec nunc cursus imperdiet. Integer feugiat, libero quis efficitur eleifend, risus justo sollicitudin dui, ac dictum ipsum libero et sapien. Quisque quam metus, gravida a elit ut, fermentum sodales nisi. Nulla aliquet dolor in massa feugiat sodales.

Cras lacinia sem at lectus pulvinar, sed iaculis justo mollis. Duis vel ligula est. Vivamus ac porta enim. Ut sit amet euismod eros. Phasellus auctor nulla venenatis ultricies pretium. Fusce eu pellentesque erat, nec eleifend elit. Vivamus volutpat lectus risus, at faucibus ipsum rutrum eget. Sed venenatis feugiat ex sed efficitur. Nulla tincidunt mollis turpis dignissim sodales.

Mauris et odio condimentum, bibendum mi at, posuere augue. Praesent posuere auctor tristique. Quisque accumsan est dui, id gravida orci iaculis eu. Praesent tincidunt tristique lorem non blandit. Mauris convallis commodo justo quis tristique. Integer ultrices quam nec est efficitur scelerisque nec et mauris. Suspendisse potenti. Proin pharetra posuere viverra.

% \section{DYCOS}
% \input{DYCOS-Algorithmus}

% \section{Analyse des DYCOS-Algorithmus}
% \input{Analyse}

% \section{Probleme des DYCOS-Algorithmus}
% \input{SchwaechenVerbesserungen}

\section{Ausblick}
Lorem ipsum dolor sit amet, consectetur adipiscing elit. Aliquam fringilla risus ut purus tincidunt sodales ac ut est. Ut eget tempus eros. Interdum et malesuada fames ac ante ipsum primis in faucibus. Sed et tempor nisl, et semper sapien. Curabitur non ex mi. Donec nec purus vel dolor porttitor rutrum malesuada sed mi. Morbi mollis tincidunt pulvinar. Pellentesque suscipit lacus magna, id cursus felis gravida ut. Vivamus lobortis dictum rhoncus. Proin tincidunt volutpat tortor, vel volutpat turpis porttitor eu. Praesent id risus vitae orci auctor suscipit.

Donec nec laoreet nisi. In tellus diam, euismod a molestie non, tincidunt in risus. Curabitur condimentum accumsan arcu, nec accumsan libero mollis congue. Praesent ac sollicitudin ligula, a pulvinar sapien. Proin eu auctor lacus. Donec et leo porttitor, pretium lacus eget, facilisis mauris. Quisque ut faucibus dui. Maecenas tempor urna erat, in mattis mauris ullamcorper ac. Cras facilisis id est vitae auctor. Phasellus sollicitudin erat neque, ac molestie justo venenatis a. Cras velit magna, tincidunt ut condimentum id, commodo id ipsum. Ut aliquam tempus elit ac malesuada. Ut pellentesque est nec nunc cursus imperdiet. Integer feugiat, libero quis efficitur eleifend, risus justo sollicitudin dui, ac dictum ipsum libero et sapien. Quisque quam metus, gravida a elit ut, fermentum sodales nisi. Nulla aliquet dolor in massa feugiat sodales.

Cras lacinia sem at lectus pulvinar, sed iaculis justo mollis. Duis vel ligula est. Vivamus ac porta enim. Ut sit amet euismod eros. Phasellus auctor nulla venenatis ultricies pretium. Fusce eu pellentesque erat, nec eleifend elit. Vivamus volutpat lectus risus, at faucibus ipsum rutrum eget. Sed venenatis feugiat ex sed efficitur. Nulla tincidunt mollis turpis dignissim sodales.

Mauris et odio condimentum, bibendum mi at, posuere augue. Praesent posuere auctor tristique. Quisque accumsan est dui, id gravida orci iaculis eu. Praesent tincidunt tristique lorem non blandit. Mauris convallis commodo justo quis tristique. Integer ultrices quam nec est efficitur scelerisque nec et mauris. Suspendisse potenti. Proin pharetra posuere viverra.

% Normaler LNCS Zitierstil
%\bibliographystyle{splncs}
\bibliographystyle{itmalpha}
% TODO: Ändern der folgenden Zeile, damit die .bib-Datei gefunden wird
\bibliography{literatur}

\end{document}

